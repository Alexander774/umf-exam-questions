\documentclass[../main.tex]{subfiles}
\begin{document}

\section[Приведение уравнений 2 порядка к каноническому виду]{Приведение к каноническому виду в точке дифференциальных уравнений в частных производных (ДУЧП) 2 порядка в $\R^n$ с линейной старшей частью. Классификация уравнений. Приведение уравнений 2 порядка к каноническому виду на плоскости.}

% Затехал: К.А.А., 372
\phantomsection
\addcontentsline{toc}{subsection}{Приведение к каноническому виду в точке из \texorpdfstring{$\R^n$}{R\textasciicircum n}}
Пусть $\Omega \subset \R^{n}.$
ДУЧП 2 порядка с линейной старшей частью:
$$\sum_{i,j=1}^n a_{ij}(x)\frac{\partial^2 u}{\partial x_i\partial x_j} + F(x, u, \nabla u) = 0; \qquad u(x)\in C^2(\Omega);\quad a_{ij} (x) \in C(\Omega)$$
Считаем $a_{ij}(x) = a_{ji}(x)$, что не сужает класса, т.к. $u_{{x_i}{x_j}} = u_{{x_j}{x_i}}$.
Хотим сделать замену так, чтобы все смешанные частные производные обратились в 0. В точке это сделать можно.

Возьмём преобразование \[y = y(x) = \begin{cases} y_1 = y_1(x_1,\dots, x_n) \\ \dots \\ y_{n} = y_{n}(x_1,\dots,x_n)\end{cases} \in C^2(U(x^0)),\quad y^0 = y(x^0);\quad U(x^0) \rightarrow V(y^0)\] (диффеоморфизм класса $C^2$ окрестности $U(x^0)$ на $V(y^0)$)

Будем предполагать $\exists$ обратного: $x = x(y)$.
% У диффеоморфизма вроде обратное преобразование $exists$ по определению...
Наша функция: $u = u(x_1 \dots x_n)$. 

Введём $\hat{u}(y) \coloneqq u[x(y)] \in C^2(V(y^0))$.
Производные: $$\pd{u}{x_i} = \sum_{k=1}^n\pd{\hat{u}}{y_k}\pd{y_k}{x_i}; \qquad \frac{\partial^2 u}{\partial x_i\partial x_j} = \sum_{k,l=1}^n\frac{\partial^2 \hat{u}}{\partial y_k\partial y_l}\pd{y_k}{x_i}\pd{y_l}{x_j} + \underbrace{\sum_{k=1}^n\pd{\hat{u}}{y_k} \frac{\partial^2 y_k}{\partial x_i\partial x_j}}_{\substack{\text{уйдёт в }\hat{F}(y, \hat{u}, \nabla_y\hat{u})}}$$
Подставляем: $$\sum_{i,j=1}^na_{ij}(x(y))\sum_{k,l=1}^n \frac{\partial^2 \hat{u}}{\partial y_k\partial y_l}\pd{y_k}{x_i}\pd{y_l}{x_j} + \hat{F}(y, \hat{u}, \nabla_y\hat{u}) = 0$$
$$\sum_{k,l=1}^n\left[\sum_{i,j=1}^na_{ij}(x(y))\pd{y_k}{x_i}\pd{y_l}{x_j}\right] \frac{\partial^2 \hat{u}}{\partial y_k\partial y_l} + \hat{F}(y, \hat{u}, \nabla_y\hat{u}) = 0,$$
$$\hat{a}_{kl}(y) \coloneqq \sum_{i,j=1}^na_{ij}(x(y))\pd{y_k}{x_i}\pd{y_l}{x_j}$$

Введём матрицы: $A(x^0) = \norm{a_{ij}(x^0)}_{i,j=1}^n; \qquad \hat{A}(y^0) = \norm{\hat{a}_{ij}(y^0)}_{i,j=1}^n.$

$J(x^0) = \norm{\pd{y_i}{x_j}(x^0)}_{i,j=1}^n$ -- в малой $U(x^0)$ задаёт преобразование $\hat{A}(y^0) = J(x^0)A(x^0)J^T(x^0)$ 

$A = A^T \Rightarrow \hat{A}^T = \hat{A}$. Вопрос в выборе $J$ такого, что $\hat{A}$ диагональна.

Пусть в $\R^n$ заданы элемент $h$ и квадратичная форма $\Phi(h).$

Введём 2 базиса: $\begin{array}{c} \left(e_1 \dots e_n\right) \\ \left(e'_1 \dots e'_n\right) \end{array}$ В них $h \sim \begin{array}{c} \xi = (\xi_1 \dots \xi_n)^T \\ \eta = (\eta_1 \dots \eta_n)^T \end{array}; \quad \Phi \sim \begin{array}{c} \norm{c_{ij}} \\ \norm{\hat{c}_{ij}}
\end{array}; \quad \Phi(h) = \begin{array}{c}
\xi^T C \xi \\ \eta^T \hat{C} \eta
\end{array}$

Пусть $\xi = S\eta$. Тогда $\Phi(h) = \eta^TS^TCS\eta = \eta^T\hat{C}\eta \ \rightarrow\ \hat{C} = S^TCS.$

Существует такой базис, что

$\hat{C} = \operatorname{diag}(\underbrace{+1, +1 \dots +1}_{p\text{ штук}}, \underbrace{-1, -1 \dots -1}_{q\text{ штук}}, 0, 0 \dots 0)$

$\Phi(h) = \eta^2_1 + \dots + \eta^2_p - \eta^2_{p+1} - \dots - \eta^2_{p+q}$

В равенстве $\hat{A}(y^0) = J(x^0)A(x^0)J^T(x^0)$ нужно взять $J(x^0) = S^T$

Такие преобразования существуют, их много. Например, $y = y^0 + J(x^0)(x-x^0)$

В этих переменных уравнение принимает вид:

$$\pdd{\hat{u}}{y_1} + \dots + \pdd{\hat{u}}{y_p} - \pdd{\hat{u}}{y_{p+1}} - \dots - \pdd{\hat{u}}{y_{p+q}} + \hat{F}(y, \hat{u}, \nabla_y\hat{u}) = 0.$$

\Subsection{Классификация уравнений}
\begin{enumerate}
	\item \textit{Эллиптический тип}: $p = n$ или $q = n$
	\item \textit{Ультрагиперболический тип}: $p+q = n$
	\item \textit{Гиперболический тип}: $p = 1,\ q = n-1$
	\item \textit{Ультрапараболический тип}: $p + q < n$
	\item \textit{Параболический тип}: $q = 0,\ p = n-1$
\end{enumerate}

\begin{remark} $\hat{A}(y^0) = J(x^0)A(x^0)J^T(x^0) \ \Rightarrow\ \sign\det\bigl( \hat{A}(y^0) \bigr) = \sign\det\bigl( A(x^0) \bigr)$ \end{remark}

В случае $n = 2$ тип уравнения в точке определяется по знаку определителя:
\begin{enumerate}
	\item \textit{Эллиптический: } $\hat{A}(y^0) = \begin{pmatrix}
		\pm1 & 0 \\ 0 & \pm1
		\end{pmatrix} \rightarrow \abs{\hat{A}(y^0)} = 1$

	\item \textit{Гиперболический:} $\hat{A}(y^0) = \begin{pmatrix}
		1 & 0 \\ 0 & -1
		\end{pmatrix} \rightarrow \abs{\hat{A}(y^0)} = -1$

	\item \textit{Параболический:}\; $\abs{\hat{A}(y^0)} = 0$.
	
\end{enumerate}

\Subsection[Приведение к каноническому виду в области из \texorpdfstring{$\R^2$}{R\textasciicircum 2}]{Приведение уравнения 2 порядка к каноническому виду на плоскости}

Рассмотрим в $\R^2$ уравнение $a(x, y)u_{xx}+2b(x,y)u_{xy}+c(x,y)u_{yy} + F(x, y, u, \nabla u) = 0$

Для определения в точке используем $d =  \left| \begin{array}{cc}
a & b \\
b & c  \end{array} \right|$

Введём преобразование $ (x,y) \mapsto (\xi, \eta),\;\ \begin{cases} \xi = \xi(x, y) \\ \eta = \eta(x, y)\end{cases}$--- \ диффеоморфизм класса $C^2$.

В новых координатах \ $\hat{a}(\xi, \eta)\hat{u}_{\xi\xi} + 2\hat{b}(\xi, \eta)\hat{u}_{\xi\eta} + \hat{c}(\xi, \eta)\hat{u}_{\eta\eta} + \hat{F}(\xi,\eta,\hat{u},\nabla_{\xi\eta}\hat{u}) = 0$
%
$$\hat{A}(\xi, \eta) = \left( \begin{array}{cc}
\hat{a} & \hat{b} \\
\hat{b} & \hat{c}
\end{array}\right); \quad A = \left( \begin{array}{cc}
a & b \\
b & c
\end{array}\right); \quad \hat{A} = JAJ^T; \quad J = \left(\begin{array}{cc}
\xi_x & \xi_y \\
\eta_x & \eta_y
\end{array}\right)$$

\Subsubsection{Гиперболический случай}

Выбираем $(x^0, y^0) \in \Omega$, пусть $d(x^0, y^0) = ac - b^2 < 0$. В силу непрерывности есть $U_\eps(x^0, y^0)$, где $d < 0$.

Во всех точках этой окрестности тип -- гиперболический.

\begin{definition}
\textit{Второй канонический тип:} $\hat{u}_{\xi\eta} + \hat{F}(\xi, \eta, \hat{u}, \nabla \hat{u}) = 0$, то есть $\hat{a} \equiv \hat{c} \equiv 0$.
\end{definition}
Введём переменную $w$, которая обозначает либо $\xi$, либо $\eta$.

Запишем \textit{характеристическое уравнение:}
\begin{equation*}
	a(x, y) w^{2}_x + 2b(x, y)w_x w_y + c(x, y)w_y^2  = 0, 
\end{equation*}
(При $w = \xi$ это выражение равно $\hat a_{11}$, а при $w = \eta$ оно равно $\hat a_{22}$)

От решений хотим $\grad w \neq 0$, так как если $\nabla \eta = 0$ или $\nabla \xi = 0$, то $\det(J) = 0$.
\begin{enumerate}[label=\asbuk*),ref=\asbuk*]
\item Пусть $a(x^0, y^0) \neq 0$, для $c(x^0, y^0) \neq 0$ рассуждения такие же.

В окрестности, где $a(x, y) \neq 0$, делим:
\begin{equation*}
	w_x^2 + \dfrac{2b}{a}w_x w_y + \dfrac{c}{a}w_y^2 = \roundBr{w_x + \dfrac{b}{a}w_y}^2 - \dfrac{b^2 - ac}{a}w_y 
	= \bigl[ w_x + \lambda_{+}(x, y)w_y \bigr] \cdot \bigl[ w_x + \lambda_{-}w_y \bigr] = 0,
\end{equation*}
где введены обозначения $\lambda_\pm = \frac{1}{a}(b \pm \sqrt{b^2 - ac})$; верно, что $\lambda_{-}  \neq \lambda_{+} \Forall (x, y) \in U_{\eps}(x^0, y^0)$, так как $d = ac - b^2 < 0$.

Достаточно занулить одну из скобок: $\ w_x + \lambda_\pm w_y = 0.$

Подобные уравнения 1 порядка решаются при помощи первых интегралов:
$$a_1(x, y)w_x + a_2(x, y)w_y = 0 
\quad\Leftrightarrow\quad
\text{решение -- ПИ системы: }
\begin{cases}
	\frac{dx}{dt} = a_1 \\
	\frac{dy}{dt} = a_2
\end{cases}
\Leftrightarrow\quad
\frac{dx}{a_1} = dt = \frac{dy}{a_2}
$$
В нашем случае $\frac{dx}{1} = \frac{dy}{\lambda_\pm} \ \Leftrightarrow\ dy - \lambda_\pm dx = 0 $ --- первые интегралы этого уравнения дают решение исходного уравнения характеристик. Если $a,b,c \in C^2$, то $\lambda_\pm$ и первые интегралы также принадлежат $C^2$.

Покажем невырожденность:
\begin{equation*}
	\begin{cases}
		\xi_x + \lambda_{+} \xi_y = 0, \\
		\eta_x + \lambda_{-} \eta_y = 0.
	\end{cases}
\end{equation*}
\begin{multline*}
	\bigr|J(x^0, y^0) \bigr| = \det
	\begin{pmatrix}
		\xi_x & \xi_y \\
		\eta_x & \eta_y 
	\end{pmatrix} = \det
	\begin{pmatrix}
		-\lambda_{+} \xi_y & \xi_y \\
		-\lambda_{-}\eta_y & \eta_y
	\end{pmatrix} = \overbrace{(\lambda_{-} - \lambda_{+})}^{\substack{\neq 0\;\text{\tiny в силу} \\ \text{\tiny гиперболичности}}}\cdot \; \xi_y \eta_y \neq 0 \\ 
	(\text{если } \xi_{y} = 0, \text{ то } \xi_x = 0 
	\ \Rightarrow\; \nabla \xi = 0)
\end{multline*}

Итак, $(x,y) \mapsto (\xi,\eta)$ -- диффеоморфизм класса $C^2$. Он зануляет $\hat{a}$ и $\hat{c}$. Получается уравнение во второй канонической форме.
\begin{remark} 
От II канонической форме к I:
\begin{equation*}
	\begin{cases}
		\alpha = \xi + \eta, \\
		\beta = \xi - \eta
	\end{cases} \Rightarrow
	\hat{u}(\xi, \eta) = \tilde{u}(\underbrace{\xi + \eta}_{\alpha}, \underbrace{\xi - \eta}_{\beta}),\; \hat{u}_{\xi} = \tilde{u}_{\alpha} +\tilde{u}_{\beta},\; u_{\xi \eta} = \tilde{u}_{\alpha \alpha} - \tilde{u}_{\beta \beta}
\end{equation*}
Тогда наше уравнение: 
\begin{equation*}
	\tilde{u}_{\alpha \alpha} - \tilde{u}_{\beta \beta} + \tilde{F}(\alpha, \beta, \tilde{u}, \nabla_{\alpha \beta}\tilde{u}) = 0 \text{ --- I каноническая форма}
\end{equation*}
\end{remark}
\item Если $a(x, y) \equiv c(x, y) \equiv 0 \Forall (x, y) \in U(x^0, y^0)$, то $b \neq 0$, иначе уравнение в нуле функции $b$ -- не второго порядка.

То есть уравнение уже имеет II каноническую форму, преобразование в I -- выше. 
\item Если $a \not\equiv 0$ или $c \not\equiv 0$, но $a(x^0, y^0) = c(x^0, y^0) = 0$, то аналогично $b(x^0, y^0) \neq 0$. \\
Заменим $\xi = x + y,\ \eta = x - y$ и получим:
$$
\begin{aligned}
\hat{a}(\xi^0, \eta^0) &= 2b(x^0, y^0) \neq 0 \\
\hat{c}(\xi^0, \eta^0) &= -2b(x^0, y^0)
\end{aligned}
$$
Это случай (а).
\end{enumerate}

\Subsubsection{Параболический случай}
Пусть в точке и некоторой ее окрестности тип параболический, то есть $ac - b^2 \equiv 0$. Ни в одной точке $a$ и $c$ не равны нулю одновременно, поскольку иначе $b = 0$ и это уравненение 1 порядка. Пусть для определённости $a \neq 0$. Уравнение характеристик:
$$
aw_x^2 + 2bw_x w_y+cw_y^2 = 0 
\quad\Leftrightarrow\quad 
(w_x + \lambda w_y)^2 = 0, 
\text{ где } \lambda =\frac{b}{a}
$$
Находим решение $w = \eta(x, y) \in C^2(U(x^0, y^0)):\ \nabla w(x, y) \ne 0$.
$$
\begin{cases}
	\xi = w & \text{эту построили} \\
  \eta = \eta(x,y) & \text{эту выбираем произвольно, чтобы был диффеоморфизм} \\
	& \text{доказательство того, что всегда можно выбрать, опущено.}
\end{cases}
$$
Мы взяли $\xi$ такое, что $\hat c \equiv 0$. Покажем, что и $\hat b \equiv 0$:
$$
\hat A = JAJ^\intercal \Rightarrow \abs{\hat A} = \abs{J}^2\cdot\abs{A} = \abs{J}^2(ac - b^2) = 0 = \underbrace{\hat a \hat c}_{=0} - \hat b^2 \Rightarrow \hat b = 0
$$
Пришли к 
$$
\hat a(\xi, \eta)\hat u_{\xi\xi} + \hat F(\xi, \eta, \hat u, \nabla_{\xi, \eta}\hat u) = 0,
$$
где $\hat a \neq 0$, иначе 1 порядок, а обратная замена дает второй порядок.

\Subsubsection{Эллиптический случай}
$$
d(x, y) = ac - b^2 > 0 \;\Forall (x, y) \in U(x^0, y^0)
$$
Во всех точках окрестности выполнено $a\ne 0$ и $c\ne 0$, иначе было бы $d = -b^2 \leq 0$.\\ 
Характеристическое уравнение:
$$
\squareBr{w_x + \lambda_+w_y}\squareBr{w_x + \lambda_-w_y} = 0,
$$
где $\lambda_\pm = \frac{b^2\pm\sqrt{b^2 - ac}}{a} = \mu \pm i\nu; \;\ \mu, \nu \in C^2$.\quad При этом для функций $\lambda_+, \lambda_-$ известно $\lambda_+ = \overline{\lambda_-}.$
\vspace{0.3em}

Получилось два линейных ДУЧП первого порядка: $$w_x \pm \lambda_\pm w_y = 0 $$
Представим $w = \xi \pm i\eta \;\ \Rightarrow\;\ 
(\xi \pm i\eta)_x + (\mu \pm i\nu)(\xi \pm i\eta)_y = 0$
$$
\begin{cases}
    \mathrm{Re:} & \xi_x + \mu\xi_y - \nu\eta_y = 0\\
    \mathrm{Im:} & \eta_x + \nu\xi_y + \mu\eta_y = 0
\end{cases}\qquad \text{Искомая замена }
\begin{cases}
    \xi = \xi(x, y) \\
    \eta = \eta(x, y)
\end{cases} \in C^2,
$$
Невырожденность:
$$ J = 
\begin{vmatrix}
    \xi_x & \xi_y \\
    \eta_x & \eta_y
\end{vmatrix} =
\begin{vmatrix}
    (-\mu\xi_y + \nu\eta_y) & \xi_y \\
    (-\nu\xi_y - \mu\eta_y) & \eta_y
\end{vmatrix} =
\nu (\xi_y^2 + \eta_y^2) \neq 0, \text{ т.к} $$
\begin{itemize}[nolistsep]
    \item $\nu \neq 0$
    \item если $\eta_y = \xi_y = 0$, то в силу уравнений на действительную и мнимую части
    
    $\xi_x = \eta_x = 0 \; \Rightarrow \; \nabla w = 0$
\end{itemize}
\begin{multline*}
a w_x^2 + 2bw_x w_y + c w_y^2 = 0 \Rightarrow \\
\underbrace{(a\xi^2_x + 2b\xi_x\xi_y + c\xi_y^2)}_{\hat a} -
\underbrace{(a\eta_x^2 + 2b\eta_x\eta_y + c\eta_y^2)}_{\hat c}+ 
\underbrace{2i(a\xi_x \eta_x + b(\xi_x\eta_y + \xi_y\eta_x) + c\xi_y\eta_y)}_{2\hat b} = 0
\end{multline*}
Откуда $\hat a = \hat c,\;\ \hat b = 0$, то есть получаем уравнение
$$
\hat u_{\xi\xi} + \hat u_{\eta\eta} + \hat F(\xi, \eta, \hat u, \nabla_{\xi\eta}\hat u) = 0
$$

\Subsection{Формальный вид уравнения характеристик}
$$
a\;dy\;dy - 2b\;dy\;dx + c\;dx\;dx =0
$$
Оно так выглядит из $(dy - \lambda_+dx)(dy - \lambda_-dx)=0$, то есть 
$$
(dy)^2 - \underbrace{(\lambda_+ + \lambda_-)}_{2b/a}\;dx\;dy + \underbrace{\lambda_+\lambda_-}_{c/a}(dx)^2 = 0
$$
\end{document}