\documentclass[../main.tex]{subfiles}
\begin{document}

\section{Билет 1. Приведение к каноническому виду в точке дифференциальных уравнений в частных производных (ДУЧП) 2 порядка в \texorpdfstring{$\R^n$}{R\textasciicircum n} с линейной старшей частью. Классификация уравнений. Приведение уравнений 2 порядка к каноническому виду на плоскости}

% Затехал: К.А.А., 372

Пусть $\Omega \subset \R^{n}.$
ДУЧП 2 порядка с линейной старшей частью:
$$\sum_{i,j=1}^n a_{ij}(x)\frac{\partial^2 u}{\partial x_i\partial x_j} + F(x, u, \nabla u) = 0; \qquad u(x)\in C^2(\Omega);\quad a_{ij} (x) \in C(\Omega)$$
Считаем $a_{ij}(x) = a_{ji}(x)$, что не сужает класса, т.к. $u_{{x_i}{x_j}} = u_{{x_j}{x_i}}$.
Хотим сделать замену так, чтобы все смешанные частные производные обратились в 0. В точке это сделать можно.

Возьмём преобразование \[y = y(x) = \begin{cases} y_1 = y_1(x_1,\dots, x_n) \\ \dots \\ y_{n} = y_{n}(x_1,\dots,x_n)\end{cases} \in C^2(U(x^0)),\quad y^0 = y(x^0);\quad U(x^0) \rightarrow V(y^0)\] (диффеоморфизм класса $C^2$ окрестности $U(x^0)$ на $V(y^0)$)

Будем предполагать $\exists$ обратного: $x = x(y)$.
% У диффеоморфизма вроде обратное преобразование $exists$ по определению...
Наша функция: $u = u(x_1 \dots x_n)$. 

Введём $\hat{u}(y) \coloneqq u[x(y)] \in C^2(V(y^0))$.
Производные: $$\pd{u}{x_i} = \sum_{k=1}^n\pd{\hat{u}}{y_k}\pd{y_k}{x_i}; \qquad \frac{\partial^2 u}{\partial x_i\partial x_j} = \sum_{k,l=1}^n\frac{\partial^2 \hat{u}}{\partial y_k\partial y_l}\pd{y_k}{x_i}\pd{y_l}{x_j} + \underbrace{\sum_{k=1}^n\pd{\hat{u}}{y_k} \frac{\partial^2 y_k}{\partial x_i\partial x_j}}_{\substack{\text{уйдёт в }\hat{F}(y, \hat{u}, \nabla_y\hat{u})}}$$
Подставляем: $$\sum_{i,j=1}^na_{ij}(x(y))\sum_{k,l=1}^n \frac{\partial^2 \hat{u}}{\partial y_k\partial y_l}\pd{y_k}{x_i}\pd{y_l}{x_j} + \hat{F}(y, \hat{u}, \nabla_y\hat{u}) = 0$$
$$\sum_{k,l=1}^n\left[\sum_{i,j=1}^na_{ij}(x(y))\pd{y_k}{x_i}\pd{y_l}{x_j}\right] \frac{\partial^2 \hat{u}}{\partial y_k\partial y_l} + \hat{F}(y, \hat{u}, \nabla_y\hat{u}) = 0,$$
$$\hat{a}_{kl}(y) \coloneqq \sum_{i,j=1}^na_{ij}(x(y))\pd{y_k}{x_i}\pd{y_l}{x_j}$$

Введём матрицы: $A(x^0) = \norm{a_{ij}(x^0)}_{i,j=1}^n; \qquad \hat{A}(y^0) = \norm{\hat{a}_{ij}(y^0)}_{i,j=1}^n.$

$J(x^0) = \norm{\pd{y_i}{x_j}(x^0)}_{i,j=1}^n$ -- в малой $U(x^0)$ задаёт преобразование $\hat{A}(y^0) = J(x^0)A(x^0)J^T(x^0)$ 

$A = A^T \Rightarrow \hat{A}^T = \hat{A}$. Вопрос в выборе $J$ такого, что $\hat{A}$ диагональна.

Пусть в $\R^n$ заданы элемент h и квадратичная форма $\Phi(h).$

Введём 2 базиса: $\begin{array}{c} \left(e_1 \dots e_n\right) \\ \left(e'_1 \dots e'_n\right) \end{array}$ В них $h \sim \begin{array}{c} \xi = (\xi_1 \dots \xi_n)^T \\ \eta = (\eta_1 \dots \eta_n)^T \end{array}; \quad \Phi \sim \begin{array}{c} \norm{c_{ij}} \\ \norm{\hat{c}_{ij}}
\end{array}; \quad \Phi(h) = \begin{array}{c}
\xi^T c \xi \\ \eta^T \hat{c} \eta
\end{array}$

Пусть $\xi = S\eta$. Тогда $\Phi(h) = \eta^TS^TCS\eta = \eta^T\hat{C}\eta \rightarrow \hat{c} = S^TCS.$

Существует такой базис, что

$\hat{C} = diag(\underbrace{+1, +1 \dots +1}_{p\text{ штук}}, \underbrace{-1, -1 \dots -1}_{q\text{ штук}}, 0, 0 \dots 0)$

$\Phi(h) = \eta^2_1 + \dots + \eta^2_p - \eta^2_{p+1} - \dots - \eta^2_{p+q}$

В равенстве $\hat{A}(y^0) = J(x^0)A(x^0)J^T(x^0)$ нужно взять $J(x^0) = S^T$

Такие преобразования существуют, их много. Например, $y = y^0 + J(x^0)(x-x^0)$

В этих переменных уравнение принимает вид:

$$\pdd{\hat{u}}{y_1} + \dots + \pdd{\hat{u}}{y_p} - \pdd{\hat{u}}{y_{p+1}} - \dots - \pdd{\hat{u}}{y_{p+q}} + \hat{F}(y, \hat{u}, \nabla_y\hat{u}) = 0.$$

\textbf{Классификация уравнений:}
\begin{enumerate}
	\item \textit{Эллиптический тип}: $p = n$ или $q = n$
	\item \textit{Ультрагиперболический тип}: $p+q = n$
	\item \textit{Гиперболический тип}: $p = 1, q = n-1$
	\item \textit{Ультрапараболический тип}: $p + q < n$
	\item \textit{Параболический тип}: $q = 0, p = n-1$
\end{enumerate}

\begin{remark} $\hat{A}(y^0) = J(x^0)A(x^0)J^T(x^0) \Rightarrow \sign\abs{\hat{A}(y^0)} = \sign\abs{A(x^0)}$ \end{remark}

В случае $n = 2$ тип уравнения в точке определяется по знаку определителя:
\begin{enumerate}
	\item \textit{Эллиптический: } $\hat{A}(y^0) = \begin{pmatrix}
		\pm1 & 0 \\ 0 & \pm1
		\end{pmatrix} \rightarrow \abs{\hat{A}(y^0)} = 1$

	\item \textit{Гиперболический:} $\hat{A}(y^0) = \begin{pmatrix}
		1 & 0 \\ 0 & -1
		\end{pmatrix} \rightarrow \abs{\hat{A}(y^0)} = -1$

	\item \textit{Параболический:}\; $\abs{\hat{A}(y^0)} = 0$.
	
\end{enumerate}

\textbf{Приведение уравнения 2 порядка к каноническому виду на плоскости:}

Рассмотрим в $\R^2$ уравнение $a(x, y)u_{xx}+2b(x,y)u_{xy}+c(x,y)u_{yy} + F(x, y, u, \nabla u) = 0$

Для определения в точке используем $d =  \left| \begin{array}{cc}
a & b \\
b & c  \end{array} \right|$

Введём преобразование $y = y(x) = \begin{cases} \xi = \xi(x, y) \\ \eta = \eta(x, y)\end{cases}$ --- диффеоморфизм класса $C^2$.

В новых координатах $\hat{a}\left(\xi, \eta\right)\hat{u}_{\xi\xi} + 2\hat{b}\left(\xi, \eta\right)\hat{u}_{\xi\eta} + \hat{c}\left(\xi, \eta\right)\hat{u}_{\eta\eta} + \hat{F}\left(\xi,\eta,\hat{u},\nabla_{\xi\eta}\hat{u}\right) = 0$
$$\hat{A}\left(\xi,\eta\right) = \left( \begin{array}{cc}
\hat{a} & \hat{b} \\
\hat{b} & \hat{c}
\end{array}\right); \quad A = \left( \begin{array}{cc}
a & b \\
b & c
\end{array}\right); \quad \hat{A} = JAJ^T; \quad J = \left(\begin{array}{cc}
\xi_x & \xi_y \\
\eta_x & \eta_y
\end{array}\right)$$

\subsection{Гиперболический случай}

Выбираем $(x_0, y_0) \in \Omega$, пусть $d(x_0, y_0) = ac - b^2 < 0$. В силу непрерывности есть $U_\epsilon(x_0, y_0)$, где $d < 0$.

Во всех точках этой окрестности тип -- гиперболический.

\begin{definition}
\textit{Второй канонический тип:} $\hat{u}_{\xi\eta} + \hat{F}(\xi, \eta, \hat{u}, \nabla \hat{u}) = 0$, то есть $\hat{a} \equiv \hat{c} \equiv 0\; \forall \xi, \eta \in V(\xi^{\circ}, \eta^{\circ})$.
\end{definition}
Введём переменную $w$, которая обозначает либо $\xi$, либо $\eta$.

Запишем \textit{характеристическое уравнение:}
\begin{equation*}
	a(x, y) w^{2}_x + 2b(x, y)w_x w_y + c(x, y)w_y^2  = 0, 
\end{equation*}
(При $w = \xi$ это выражение равно $\hat a_{11}$, а при $w = \eta$ оно равно $\hat a_{22}$)

От решений хотим $\text{grad}\;w \neq 0$, так как если $\nabla \eta = 0$ или $\nabla \xi = 0$, то $J = 0$.

\begin{remark}
	$w(x) = 0$ -- характеристическая $\Rightarrow \tilde{w}(x) =  w(x) - c = 0$ -- также характеристическая: $\tilde{c} \in C^2,\; \nabla \tilde{w} \neq 0$, $\tilde{w}$ -- удовлетворяет характеристическому уравнению.
\end{remark}

\begin{definition}
Переменные $\xi, \eta$ - \textit{характеристические}; поверхности $\xi = C_1,\; \eta = C_2$ - \textit{характеристические}.
\end{definition} 
\begin{enumerate}[label=\asbuk*),ref=\asbuk*]
\item Пусть $a(x^{\circ}, y^{\circ}) \neq 0$, для $c(x^{\circ}, y^{\circ}) \neq 0$ рассуждения такие же.

В окрестности, где $a(x, y) \neq 0\; \brbr{u_{\eps}(x^{\circ}, y^{\circ})}$, делим:
\begin{equation*}
	w_x^2 + \dfrac{2b}{a}w_x w_y + \dfrac{c}{a}w_y^2 = \erbr{w_x + \dfrac{b}{a}w_y}^2 - \dfrac{b^2 - ac}{a}w_y = \bfbr{w_x + \lambda_{+}(x, y)w_y} \cdot \bfbr{w_x + \lambda_{-}w_y} = 0,
\end{equation*}
где введены обозначения $\lambda_{\pm} = \frac{1}{a}(b \pm \sqrt{b^2 - ac})$; верно, что $\lambda_{-}  \neq \lambda_{+}\; \forall (x, y) \in U_{\eps}(x^{\circ}, y^{\circ})$, так как $d = ac - b^2 < 0$.

Рассмотрим ЛДУЧП 1-го порядка $w_x + \lambda(x, y)w_y = 0$.

Из теории: у однородного ДУЧП $a_1(x, y)w_x + a_2(x, y)w_y = 0$ при условии $a_1^2 + a_2^2 > 0$ решение есть: 
\begin{equation*}
	\exists w(x, y) \in C^2(\Omega),\; \nabla w \neq 0\; \text{и}\; \dfrac{dx}{a_1} = \dfrac{dy}{a_2} -\; \text{первый интеграл}
\end{equation*}
В нашем случае $\frac{dx}{1} = \frac{dy}{\lambda} \Leftrightarrow dy - \lambda dx = 0$ --- ПИ этого уравнения даёт решение исходного ДУЧП.

Значит, в обеих скобках есть по решению, причём $\nabla w \neq 0$.

Покажем невырожденность:
\begin{equation*}
	\begin{cases}
		\xi_x + \lambda_{+} \xi_y = 0, \\
		\eta_x + \lambda_{-} \eta_y = 0.
	\end{cases}
\end{equation*}
\begin{multline*}
	\bigr|J(x^{\circ}, y^{\circ}) \bigr| = \det
	\begin{pmatrix}
		\xi_x & \xi_y \\
		\eta_x & \eta_y 
	\end{pmatrix} = \det
	\begin{pmatrix}
		-\lambda_{+} \xi_y & \xi_y \\
		-\lambda_{-}\eta_y & \eta_y
	\end{pmatrix} = \underbrace{(\lambda_{-} - \lambda_{+})}_{\substack{\neq 0\;\text{\tiny в силу} \\ \text{\tiny гиперболичности}}}\cdot \; \xi_y \eta_y \neq 0 \\[10pt] (\text{если}\; \xi_{y} = 0, \text{ то } \xi_x = 0 
	\Rightarrow \nabla \xi = 0)
\end{multline*}

Итак, $\brbr{\xi(x, y), \eta(x, y)}$ - диффеорморфизм класса $C^2$. Он зануляет $\hat{a}$ и $\hat{c}$. Получается уравнение во второй канонической форме.
\begin{remark} 
От II канонической форме к I:
\begin{equation*}
	\begin{cases}
		\alpha = \xi + \eta, \\
		\beta = \xi - \eta
	\end{cases} \Rightarrow
	\hat{u}(\xi, \eta) = \tilde{u}(\underbrace{\xi + \eta}_{\alpha}, \underbrace{\xi - \eta}_{\beta}),\; \hat{u}_{\xi} = \tilde{u}_{\alpha} +\tilde{u}_{\beta},\; u_{\xi \eta} = \tilde{u}_{\alpha \alpha} - \tilde{u}_{\beta \beta}
\end{equation*}
Тогда наше уравнение: 
\begin{equation*}
	\tilde{u}_{\alpha \alpha} - \tilde{u}_{\beta \beta} + \tilde{F}(\alpha, \beta, \tilde{u}, \nabla_{\alpha \beta}\tilde{u}) = 0 -\; \text{I каноническая форма}
\end{equation*}
\end{remark}
\item Если $a(x, y) \equiv c(x, y) \equiv 0\; \forall (x, y) \in U(x^{\circ}, y^{\circ})$, то $b \neq 0$, иначе уравнение в нуле функции $b$ -- не второго порядка.

То есть уравнение уже имеет II каноническую форму, преобразование в I -- выше. 
\item Если $a(x^0, y^0) = c(x^0, y^0) = 0$, но в любой окрестности $W(x^0, y^0)$ есть точки, где
$$
\abs{a(x^*, y^*)} + \abs{c(x^*, y^*)} > 0,
$$ то заменим $\xi = x + y,\ \eta = x - y$ и получим случай а.
\end{enumerate}
\end{document}