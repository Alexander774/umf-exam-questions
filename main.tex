% Шаблон Юрия Максимова (отредактированный)

\documentclass[12pt]{article}
\usepackage{graphicx}
\usepackage[T2A]{fontenc}
\usepackage[utf8]{inputenc}
\usepackage[english,russian]{babel}
\usepackage{amsthm}       % оформление теорем
\usepackage{amscd}        % используется в одном месте в конце 28 билета
\usepackage{amsmath}
\usepackage{esint}        % для \oiint
\usepackage{wrapfig}      % можно иллюстрации сбоку от текста
\usepackage{datetime}
\usepackage{cancel}       % чтобы зачёркивать члены в формулах
\usepackage{mathtools}
\usepackage{amssymb}
\usepackage[colorlinks = true, urlcolor = blue]{hyperref}
\usepackage[inline]{enumitem}
\usepackage{cmll}         % Используется в одном месте в 14 билете
\usepackage{varwidth}     % Чтобы сделать minipage со списком авторов
\usepackage{subfiles}

% Ivanychev theorem definitions
\theoremstyle{plain}
\newtheorem{theorem}{Теорема}[section]
\newtheorem{lemma}[theorem]{Лемма}
\newtheorem{statement}[theorem]{Утверждение}
\newtheorem*{conseq}{Следствие}
\newtheorem{offtop}{Offtop}[section]

\theoremstyle{definition}
\newtheorem{definition}{Определение}[section]
\newtheorem{example}{Пример}[section]

\theoremstyle{remark}
\newtheorem*{remark}{Замечание}


\renewcommand{\baselinestretch}{1.0}
\renewcommand\normalsize{\sloppypar}

\setlength{\topmargin}{-0.5in}
\setlength{\textheight}{9.1in}
\setlength{\oddsidemargin}{-0.3in}
\setlength{\evensidemargin}{-0.3in}
\setlength{\textwidth}{7in}
\setlength{\parindent}{0ex}
\setlength{\parskip}{1ex}



\newcommand{\while}[2]{\left. #1\right|_{#2}}
\newcommand{\brs}[1]{\left(#1\right)}
\newcommand{\sbrs}[1]{\left[#1\right]}
\newcommand{\fbrs}[1]{\left\{#1\right\}}
\newcommand{\rbrs}[1]{\left\langle #1 \right\rangle}

\newcommand{\brbr}[1]{\bigl(#1\bigr)}
\newcommand{\erbr}[1]{\biggl(#1\biggr)}
\newcommand{\bfbr}[1]{\bigl[#1\bigr]}
\newcommand{\efbr}[1]{\biggl[#1\biggr]}


\newcommand{\R}{\mathbb{R}}
\newcommand{\Q}{\mathbb{Q}}
\renewcommand{\C}{\mathbb{C}}
\newcommand{\N}{\mathbb{N}}
\newcommand{\Z}{\mathbb{Z}}

\renewcommand{\phi}{\varphi}
\newcommand{\eps}{\varepsilon}

\newcommand{\sign}{\mathrm{sign}\;}
\newcommand{\Equiv}{\Leftrightarrow}


\let \oldcmdforall \forall
\renewcommand{\forall}{\;\oldcmdforall\:}
\let \oldcmdexists \exists
\renewcommand{\exists}{\;\oldcmdexists\:}
\renewcommand{\leq}{\leqslant}
\renewcommand{\geq}{\geqslant}

% \abs and \norm
\DeclarePairedDelimiter\abs{\lvert}{\rvert}%
\DeclarePairedDelimiter\norm{\lVert}{\rVert}%
\makeatletter
\let\oldabs\abs
\def\abs{\@ifstar{\oldabs}{\oldabs*}}
%
\let\oldnorm\norm
\def\norm{\@ifstar{\oldnorm}{\oldnorm*}}
\makeatother



\newcommand{\pd}[2]{\frac{\partial #1}{\partial #2}}
%   \pd{f}{x}  ---> df/fx

\newcommand{\spd}[3]{\frac{\partial^2 #1}{\partial #2\partial #3}}
% spd{f}{x}{y} ---> (d^2 f) / (dx dy)

\newcommand{\npd}[3]{\frac{\partial^{#3} #1}{\partial #2^{#3}}}
% npd{f}{x}{n} ---> (d^n f) / dx^n

\newcommand{\pdd}[2]{\npd{#1}{#2}{2}}
%   pdd{f}{x}  ---> (d^2 f) / dx^2

% Позволяет писать \item'ы в строчку
% Пример:
% \begin{itemize}
% 	\item bla \inlineitem bla \inlineitem bla
%		== здесь перешли на новую строку ==
%	\item bla \inlineitem bla \inlineitem bla
% \end{itemize}
\makeatletter
\newcommand{\inlineitem}[1][]{%
\ifnum\enit@type=\tw@
    {\descriptionlabel{#1}}
  \hspace{\labelsep}%
\else
  \ifnum\enit@type=\z@
       \refstepcounter{\@listctr}\fi
    \quad\@itemlabel\hspace{\labelsep}%
\fi}

\makeatletter
\AddEnumerateCounter{\asbuk}{\russian@alph}{щ}
\makeatother

\usepackage{tikz}
\usetikzlibrary{patterns}

\title{Уравнения математической физики}
\date{\today \quad \currenttime}
\author{Билеты к экзамену. Поток В.И. Зубова}

\usepackage{natbib}
\usepackage{graphicx}

\begin{document}

\maketitle

Конспект подготовлен на основе лекций В.И. Зубова и подготовленных билетов Павла Останина и Михаила Христиченко. Полный список авторов приведён в конце.

Опечатки исправлять здесь: \textsl{\href{https://github.com/Batmaev/umf-exam-questions}{github.com/batmaev/umf-exam-questions}}

\tableofcontents

%\subfile{biblio.tex}

\newpage
\subfile{q1/1 part1.tex}
\subfile{q1/1 part2.tex}
\newpage
\subfile{q2/2.tex}
\newpage
\subfile{q3/3.tex}
\newpage
% Сергей Иванычев 
\subfile{q4/4.tex}
\newpage
\subfile{q5/5.tex}
\newpage
% Сергей Иванычев 
\subfile{q6/6.tex}
\newpage
\subfile{q7/7.tex}
\newpage
\subfile{q8/8.tex}
\newpage
\subfile{q9/9.tex}
\newpage
\subfile{q10/10.tex}
\newpage
\subfile{q11/11.tex}
\newpage
\subfile{q12/12.tex}
\newpage
\subfile{q13/13.tex}
\newpage
\subfile{q14/14.tex}
\newpage
\subfile{q15/15.tex}
\newpage
\subfile{q16/16.tex}
\newpage
\subfile{q17/17.tex}
\newpage
\subfile{q18/18.tex}
\newpage
\subfile{q19/19.tex}
\newpage
\subfile{q20/20.tex}
\newpage
\subfile{q21/21.tex}
\newpage
\subfile{q22/22.tex}
\newpage
\subfile{q23/23_1.tex}
\subfile{q23/23_2.tex}
\subfile{q23/23_3.tex}
\newpage
\subfile{q24/24.tex}
\newpage
\subfile{q25/25.tex}
\newpage
\subfile{q26/26.tex}
\newpage
\subfile{q27/27.tex}
\newpage
\subfile{q28/28.tex}
\newpage
\subfile{q29/29.tex}
\newpage
\subfile{q30/30cut.tex}
\newpage
\subfile{q31/31.tex}
\newpage
\subfile{q32/32.tex}

\newpage

\centering
\begin{varwidth}{\textwidth}
  \centering
  \subsection*{Делали:}
  \begin{itemize}
    \item Иванычев Сергей, 376 группа
    \item Погодин Роман, 374 группа
    \item Нагайко Иван, 372 группа
    \item Рязанов Андрей, 374 группа
    \item Федоряка Дмитрий, 374 группа
    \item Багно Богдан, 376 группа
    \item Изутин Никита, 378 группа
    \item Ермолова Марина, 373 группа
    \item Хасянов Расул, 371 группа
    \item Михальченко Егор, 371 группа
    \item Шлёнский Владислав, 374 группа
    \item Цветкова Ольга, 374 группа
    \item Молибог Игорь, 374 группа
    \item Чигринский Виктор, 374 группа
    \item Леонтьев Семён, 377 группа
    \item Кильянов Александр, 372 группа 
    \item Тернов Лёха, 228 группа
  \end{itemize}
\end{varwidth}
\end{document}
