\section{Билет 7. Формула Пуассона решения задачи Коши для волнового уравнения в $\bR^2$. Метод спуска. Диффузия волн}
% Затехал: Погодин Роман
Рассматривается задача
\begin{equation}
\label{eq::7::cauchy}
\left\{
  \begin{array}{ll}
  u_{tt}-a^2(u_{x_1x_1}+u_{x_2x_2})=f(t,x_1,x_2),\ t>0,\, (x_1, x_2)\in \bR^2\\
  \left. u\right|_{t=0}=u_0(x_1, x_2);\ \left. u_t\right|_{t=0}=u_1(x_1, x_2)
  \end{array}
\right.
\end{equation}
Используем \textbf{метод спуска}: перейдем в $\bR^3$:
\begin{equation*}
\left\{
  \begin{array}{ll}
  u_{tt}-a^2(u_{x_1x_1}+u_{x_2x_2}+u_{x_3x_3})=f(t,x_1,x_2)\\
  \left. u\right|_{t=0}=u_0(x_1, x_2);\ \left. u_t\right|_{t=0}=u_1(x_1, x_2)
  \end{array}
\right.
\end{equation*}
Для этой задачи решение мы уже знаем. Покажем, что оно не зависит от третьей переменной.
\[
u(t, x_1, x_2, x_3)=\frac{\partial}{\partial t}\left[\frac{1}{4\pi a^2t}\iint\limits_{|\xi - x|=at}u_0(\xi_1, \xi_2)dS_{\xi} \right]+\frac{1}{4\pi a^2t}\underbrace{\iint\limits_{|\xi - x|=at}u_1(\xi_1, \xi_2)dS_{\xi}}_{V(t,x)}+\frac{1}{4\pi a^2}\iiint\limits_{|\xi - x|<at}\frac{f\left(t-\frac{|\xi - x|}{a}, \xi \right)}{|\xi - x|}d\xi
\]

Покажем, например, что функция $V(t,x)$ не зависит от $x_3$.
Сфера, по которой ведется интегрирование, разбивается на две полусферы, проектирующиеся в одну окружность, т.е. $S_{at}=S_{at}^+\bigcup S_{at}^-$, причем эти полусферы задаются уравнениями $\xi_3=x_3\pm \sqrt{a^2t^2-(\xi_1 -x_1)^2-(\xi_2 -x_2)^2}=x_3\pm \sqrt{a^2t^2-|\xi '-x'|^2}$.

Т.к. интегралы $\displaystyle\iint\limits_{S_{at}^+}=\iint\limits_{S_{at}^-}$, имеем $V(t,x)=\dfrac{1}{2\pi a^2t}\displaystyle\iint\limits_{S_{at}^+}u_1(\xi_1, \xi_2)dS_{\xi}$.

Из мат. анализа известно, что для поверхности $S$, заданной явно: $\xi_3=F(\xi_1, \xi_2),\ (\xi_1, \xi_2)\in D$, справделиво 
\[
\iint\limits_S u(\xi )dS_{\xi}=\iint\limits_D u\left( \xi_1, \xi_2, F(\xi_1, \xi_2)\right)\sqrt{1+F_{\xi_1}^2+F_{\xi_2}^2}d\xi_1d\xi_2.
\]
В нашем случае $\sqrt{1+F_{\xi_1}^2+F_{\xi_2}^2}=\sqrt{1+\dfrac{(\xi_1 - x_1)^2+(\xi_2 - x_2)^2}{a^2t^2 - (\xi_1 - x_1)^2-(\xi_2 - x_2)^2}}=\dfrac{at}{\sqrt{a^2t^2 - (\xi_1 - x_1)^2-(\xi_2 - x_2)^2}}$.

Получили, что 
\[
V(t, x)=\frac{1}{2\pi a}\iint\limits_{|\xi' -x'|<at}\frac{u_1(\xi_1, \xi_2)}{\sqrt{a^2t^2-|\xi' -x'|^2}}d\xi_1d\xi_2
\]
не зависит от третьей переменной. 

Аналогично для двух других слагаемых, т.~к. во всех трех под знаками интегралов или производных можно выделить интеграл вида $\displaystyle\iint\limits_{|\xi -x|=at}\phi (\xi_1, \xi_2)dS_{\xi}$, который, как показано выше, от $x_3$ не зависит. Доказанное можно сформулировать как теорему:
\begin{theorem}
Пусть в задаче Коши \eqref{eq::7::cauchy} $u_0(x)\in C^3(\bR^2),\, u_1(x)\in C^2(\bR^2),\, D_x^{\alpha}f(t,x)\in C\{t\geq 0, x\in\bR^2\}\ \forall\alpha :\ |\alpha |\leq 2$. Тогда функция ($d\xi =d\xi_1d\xi_2$)
\[
\begin{split}
u(t, x)=&\pd{}{t}\left[\frac{1}{2\pi a}\iint\limits_{|\xi - x|<at}\frac{u_0(\xi )d\xi}{\sqrt{a^2t^2-|\xi - x|^2}} \right]+ \frac{1}{2\pi a}\iint\limits_{|\xi - x|<at}\frac{u_1(\xi )d\xi}{\sqrt{a^2t^2-|\xi - x|^2}}+\\
&\int\limits_0^t\left[\frac{1}{2\pi a}\iint\limits_{|\xi - x|<a(t-\tau )}\frac{f(\tau ,\xi )d\xi}{\sqrt{a^2(t-\tau )^2-|\xi - x|^2}} \right]d\tau
\end{split}
\]
принадлежит $C^2\{t\geq 0, x\in\bR^2 \}$ и является классическим решением задачи Коши \eqref{eq::7::cauchy}.
\end{theorem}
\begin{definition}[Диффузия волн]
-- это отсутствие принципа Гюйгенса. В $\bR^2$ его нет. Есть эффект последействия: передний фронт есть, а заднего нет, так как интегралы берутся не по контурам, а по всей внутренней области.
\end{definition}
Можно привести более наглядное доказательство. Пусть носители функций $u_0$, $u_1$ содержатся в некотором компакте $M$. Тогда <<погрузив>>~ $\R^2$ в $\R^3$, получим, что носитель начального возмущения - неограниченный цилиндр $\{(x_1, x_2, x_3): ~(x_1, x_2) \in M,~ x_3 \in \R \}$. Следовательно, начальное возмущение неограничено в пространстве, и возмущение в любой точке неограниченно во времени(у цилиндрических волн отсутствует задний фронт)








