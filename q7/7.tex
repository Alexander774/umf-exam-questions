\documentclass[../main.tex]{subfiles}
\begin{document}

\section{Формула Пуассона решения задачи Коши для волнового уравнения в \texorpdfstring{$\R^2$}{R\textasciicircum 2}. Метод спуска. Диффузия волн}
% Затехал: Погодин Роман
Рассматривается задача
\begin{equation}
\label{eq::7::cauchy}
\begin{cases}
  u_{tt}-a^2(u_{x_1x_1}+u_{x_2x_2})=f(t,x_1,x_2),\qquad t>0,\ (x_1, x_2)\in \R^2\\
  u|_{t=0}=u_0(x_1, x_2)\\ 
  u_t|_{t=0}=u_1(x_1, x_2)
\end{cases}
\end{equation}
Используем \textbf{метод спуска}: перейдем в $\R^3$:
\begin{equation*}
\begin{cases}
  u_{tt}-a^2(u_{x_1x_1}+u_{x_2x_2}+u_{x_3x_3})=f(t,x_1,x_2) \hspace{7.5em}\\
  %hspace нужно, чтобы выравнять это уравнение и предыдущее
  u|_{t=0}=u_0(x_1, x_2)\\
  u_t|_{t=0}=u_1(x_1, x_2)
\end{cases}
\end{equation*}
Для этой задачи решение мы уже знаем. Покажем, что оно не зависит от третьей переменной.
\begin{multline*}
u(t, x_1, x_2, x_3) = \\
= \pd{}{t} \frac{1}{4\pi a^2t} \oiint\limits_{|\xi-x|=at} u_0(\xi_1, \xi_2)\, dS_\xi 
+\frac{1}{4\pi a^2t}\underbrace{\oiint\limits_{|\xi-x|=at} u_1(\xi_1, \xi_2)\, dS_{\xi}}_{V(t,x)}
+\frac{1}{4\pi a^2}\iiint\limits_{|\xi - x|<at}\frac{f\left(t-\frac{|\xi - x|}{a}, \xi \right)}{|\xi - x|}\, d\xi
\end{multline*}

Покажем, например, что функция $V(t,x)$ не зависит от $x_3$. Сфера, по которой ведется интегрирование, разбивается на две полусферы, которые задаются уравнениями
$$\xi_3=x_3\pm \sqrt{a^2t^2-(\xi_1 -x_1)^2-(\xi_2 -x_2)^2}=x_3\pm \sqrt{a^2t^2-|\xi '-x'|^2}$$

Интегралы по полусферам равны, поэтому $V(t,x)=\dfrac{1}{2\pi a^2t}\displaystyle\iint\limits_{S^+}u_1(\xi_1, \xi_2)\, dS_{\xi}$

Из матанализа известно, что если поверхность $S$ задана в виде графика некоторой функции $\xi_3=F(\xi_1,\, \xi_2),\;\ (\xi_1, \xi_2)\in D$, то поверхностный интеграл можно свести к двойному по формуле
\[
\iint\limits_S u(\xi )\, dS_{\xi}=\iint\limits_D u(\xi_1,\, \xi_2,\, F(\xi_1,\, \xi_2))\, 
\sqrt{1 + \roundBr{F'_{\xi_1}}^2 + \roundBr{F'_{\xi_2}}^2}\, d\xi_1 d\xi_2
\]
В нашем случае 
$$
\sqrt{1 + \roundBr{F'_{\xi_1}}^2 + \roundBr{F'_{\xi_2}}^2}
=\sqrt{1+\dfrac{(\xi_1 - x_1)^2+(\xi_2 - x_2)^2}{a^2t^2 - (\xi_1 - x_1)^2-(\xi_2 - x_2)^2}}
=\dfrac{at}{\sqrt{a^2t^2 - (\xi_1 - x_1)^2-(\xi_2 - x_2)^2}}$$

Получили, что 
\[
V(t, x)=\frac{1}{2\pi a}\iint\limits_{|\xi' -x'|<at}\frac{u_1(\xi_1, \xi_2)}{\sqrt{a^2t^2-|\xi' -x'|^2}}\, d\xi_1d\xi_2
\]
не зависит от третьей переменной. 

Аналогично для двух других слагаемых, т.~к. во всех трех под знаками интегралов или производных можно выделить интеграл вида $\displaystyle\oiint\limits_{|\xi -x|=at}\phi (\xi_1, \xi_2)\, dS_{\xi}$,\, который, как показано выше, от $x_3$ не зависит. Доказанное можно сформулировать как теорему:

\begin{theorem}
Пусть в задаче Коши \eqref{eq::7::cauchy}:
$u_0 \in C^3(\R^2),\ 
u_1 \in C^2(\R^2),\
f \in C_{t,x}^{0,2}\, \{t\geq 0,\ x \in \R^2\}$ 
Тогда функция ($d\xi =d\xi_1d\xi_2$)
\[
\begin{split}
u(t,x) = \pd{}{t} &\left[ \frac{1}{2\pi a}
\iint\limits_{|\xi - x| < at} \frac{u_0(\xi)\, d\xi}{\sqrt{a^2t^2-|\xi - x|^2}} \right]
+ \frac{1}{2\pi a} \iint\limits_{|\xi - x|<at} \frac{u_1(\xi)\, d\xi}{\sqrt{a^2t^2-|\xi - x|^2}} +\\
\int\limits_0^t &\left[ \frac{1}{2\pi a}
\iint\limits_{|\xi - x| < a(t-\tau)} \frac{f(\tau,\xi)\, d\xi}{\sqrt{a^2(t-\tau )^2-|\xi - x|^2}} \right] d\tau
\end{split}
\]
принадлежит $C^2\{t\geq 0,\ x\in\R^2 \}$ и является классическим решением задачи Коши \eqref{eq::7::cauchy}.
\end{theorem}
\begin{definition} \textbf{\emph{Диффузия волн}}
-- это отсутствие принципа Гюйгенса. В $\R^2$ его нет. Есть эффект последействия: передний фронт есть, а заднего нет, так как интегралы берутся не по контурам, а по всей внутренней области.
\end{definition}
Можно привести более наглядное доказательство. Пусть носители функций $u_0$, $u_1$ содержатся в некотором компакте $M$. Тогда, <<погрузив>>~ $\R^2$ в $\R^3$, получим, что носитель начального возмущения - неограниченный цилиндр $\{(x_1, x_2, x_3)\colon ~(x_1, x_2) \in M,~ x_3 \in \R \}$. Следовательно, начальное возмущение не ограничено в пространстве, и возмущение в любой точке не ограничено во времени (у волн, cозданных цилиндром, отсутствует задний фронт)

\end{document}