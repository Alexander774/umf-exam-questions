\section{Симметричность и положительная определенность оператора $-\Delta$ при однородном граничном условии Дирихле. Положительность собственных значений и ортогональность собственных функций.}
% Затехала Ермолова Марина
Задача на собственные функции и собственные значения оператора Лапласа при однородном условии Дирихле:
Найти $\lambda$ и $u(x) \in C^1(\overline{\Omega}) \cap C^2(\Omega)$, где $\Omega$ - область с кусочно-гладкой границей $\Gamma$, такие, что
\[
\begin{cases}
&- \Delta u= \lambda u,\\
& \while{u}\Gamma=0,\\
& u(x) \neq 0,
\end{cases} 
\]

\begin{statement}[без доказательства]
Существует счетное число собственных значений $\{\lambda_k\}_k,  \\ \{\lambda_k\} \to \infty$, причем каждому $\lambda_k$ соответствует конечное число собственных функций.
\end{statement} 
Формула Грина справедлива и для комплексных функций. Считаем, что $\lambda \in \C$
$$
\int\limits_{\Omega} \overline{u} \Delta u dx = - \lambda \int\limits_{\Omega} u \overline{u} dx$$
$$
\int\limits_{\Omega} \overline{u} \Delta u dx = \cancelto{0}{\oint\limits_{\partial \Omega} \frac{\partial u}{\partial \vec{n}} \overline{u} dS }- \int\limits_{\Omega} (\nabla u, \nabla \overline{u}) dx$$
$\Rightarrow \lambda = \dfrac{\int\limits_{\Omega}|\nabla u|^2 dx}{\int\limits_{\Omega}|u|^2 dx}$ - соотношение Рэлея.\\
$\Rightarrow \lambda \in \mathbb{R}, \lambda >0 $ (строго больше, т.к. $\nabla u = 0 \Rightarrow u =0$). В задаче Неймана $\lambda = 0$ возможно.\\
\begin{statement} Оператор $-\Delta$ с граничными условиями Дирихле является симметричным относительно скалярного произведения в $\mathbb{L}_2(\Omega): (u,v) = \int\limits_{\Omega} u(x) \overline{v}(x)dx $ 
\end{statement}
\begin{proof}
Пусть $u(x), v(x)$ лежат в области  определения нашего оператора $$D_0(-\Delta) = \{u(x): u(x) \in C^1(\overline{\Omega}) \cap C^2(\Omega); \ \Delta u(x) \in C(\overline{\Omega}), \ \while{u}{\partial \Omega} = 0 \}.$$
Симметричность оператора означает, что $$(-\Delta u,v)=(u, -\Delta v) \Forall u,v \in D_0(-\Delta).$$ Проверим это: $$ (-\Delta u,v) - (u, -\Delta v) = \int\limits_{\Omega} (u \Delta v- v \Delta u)dx \stackrel{\text{ 2 формула Грина}}{=} \oint \limits_{\partial \Omega} ( \frac{\partial v}{\partial \vec{n}}u - \frac{\partial u}{\partial \vec{n}}v) dS =0.$$
\end{proof}
\begin{statement} Собственные функции рассматриваемого оператора $u_k(x)$ и $u_m(x)$, соответствующие различным собственным значениям $\lambda_k$ и $\lambda_m$, ортогональны относительно скалярного произведения в $\mathbb{L}_2(\Omega)$.
\end{statement}
\begin{proof}
$$(-\Delta u_k, u_m) = (u_k, -\Delta u_m)$$ $$\lambda_k (u_k, u_m)=  \lambda_m (u_k, u_m)$$ $$\Rightarrow (\lambda_k - \lambda_m) (u_k, u_m) = 0 \Rightarrow (u_k, u_m) = 0 $$
\end{proof}
 
 