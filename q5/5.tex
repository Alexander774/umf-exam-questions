\documentclass[../main.tex]{subfiles}
\begin{document}


\section{Билет 5. Формула Пуассона-Кирхгофа решения задачи Коши для однородного волнового уравнения в \texorpdfstring{$\R^3$}{R\textasciicircum 3}. Существование классического решения этой задачи.}
% Затехал: Погодин Роман
\begin{theorem}[Из курса мат. анализа]
Пусть $\Omega_x\subset\R^n,\ \Omega_y\subset\R^m$ -- ограниченные области, $f(x,y):\overline{\Omega}_x\times\overline{\Omega}_y\rightarrow \R,\ f\in C\left(\overline{\Omega}_x\times\overline{\Omega}_y \right)$. Тогда $J(y)=\displaystyle\int\limits_{\Omega_x}f(x,y)dx\in C\left(\overline{\Omega}_y \right)$.\\
Если к тому же $\dfrac{\partial f}{\partial y_k}\in C\left(\overline{\Omega}_x\times\overline{\Omega}_y \right)$, то $J(y)$ имеет непрерывную на $\overline{\Omega}_y$ частную производную $\dfrac{\partial J(y)}{\partial y_k}$, которая равна $\displaystyle\int\limits_{\Omega_x}\dfrac{\partial f}{\partial y_k}(x,y)dx.$
\end{theorem}
Обозначим ($\tau$ - вспомогательный параметр)
\[
u_g(t,x,\tau)=\frac{1}{4\pi a^2t}\displaystyle\oiint\limits_{|\xi - x|=at}g(\xi ,\tau ) dS_{\xi},\ a>0,\, t> 0,\, x\in \R^3,\, \tau \geq 0,\, \xi\in\R^3.
\]
\begin{lemma}
\label{lem:5:continuity}
Пусть $g(\xi, \tau )$ такая, что 
\begin{enumerate}
\item $g \in C\,\{\xi\in\R^3,\tau\geq 0 \}$
\item $D_{\xi}^{\alpha}g(\xi , \tau )\in C\,\{\xi\in\R^3,\tau\geq 0 \} \Forall\alpha : |\alpha |\leq p.$
\end{enumerate}
Тогда
\begin{enumerate}
\item $D_{t,x}^{\alpha}u_g(t,x,\tau )\in C\,\{\mathbf{t\geq 0}, x\in\R^3,\tau\geq 0 \} \Forall\alpha : |\alpha |\leq p$
\item $\lim\limits_{t\rightarrow +0}u_g(t,x,\tau )=0$
\item При $p\geq 1$ $\lim\limits_{t\rightarrow +0}\dfrac{\partial u_g}{\partial t}=g(x,\tau)$
\end{enumerate}
\end{lemma}
\begin{proof} Докажем отдельно все три утверждения.
\begin{enumerate}
\item Сведем интеграл к интегралу по единичной сфере с центром в нуле с помощью такой замены:
\[
\eta = \frac{\xi - x}{at}\Rightarrow \xi = x+at\eta ,\ |\vec{\eta} |=1.
\]
В таком случае элемент площади $dS_{\xi}=(at)^2dS_{\eta}$. Во введенном выше интеграле получим 
\[
u_g(t,x,\tau)=\frac{(at)^2}{4\pi a^2t}\oiint\limits_{|\eta |=1}g(x+at\eta ,\tau ) dS_{\eta}=tJ_g(t,x,\tau ),
\]
где $\displaystyle J_g(t,x,\tau )= \frac{1}{4\pi}\oiint\limits_{|\eta |=1}g(x+at\eta ,\tau ) dS_{\eta}$

Теперь интеграл уже по фиксированному множеству. Все выкладки справедливы при $t>0$. Функция
\[
\overline{g}(t,x,\eta ,\tau)=g(x+at\eta, \tau)\in C\,\{t\geq 0, x\in \R^3 , |\eta | = 1, \tau\geq 0\}.
\]
Тогда по теореме из начала билета $J_g(t,x,\tau )\in C\,\{t\geq 0, x\in \R^3,\tau\geq 0\}.$

Аналогично будет для производных в силу второй части той же теоремы и наличия соответствующих производных у функции $g$.

\item $\lim\limits_{t\rightarrow +0}U_g(t,x,\tau )=\lim\limits_{t\rightarrow +0}tJ_g(t,x,\tau )=\lim\limits_{t\rightarrow +0}t\cdot\lim\limits_{t\rightarrow +0}J_g(t,x,\tau ) = 0$ (последний предел конечен в силу непрерывности).

Можно записать
\[
U_g(t,x,\tau ) = \begin{cases}
u_g(t,x, \tau ),\ t>0\\
0,\ t=0
\end{cases} \in C\left(\overline{\Omega} \right),
\]
где последнее включение означает непрерывные в области и непрерывно продолжимые на границу функции.

\item При $p\geq 1$ запишем следующее:
\begin{equation*}
\begin{split}
&\lim\limits_{t\rightarrow +0}\frac{\partial}{\partial t} u_g(t,x, \tau) = \lim\limits_{t\rightarrow +0}\frac{\partial}{\partial t}\left[tJ_g(t,x,\tau )\right]=\lim\limits_{t\rightarrow +0}J_g(t,x,\tau )+\cancelto{0}{\lim\limits_{t\rightarrow +0}t\cdot \frac{\partial}{\partial t}J_g(t,x,\tau )}=\\
&=J_g(0, x, \tau )=\frac{1}{4\pi}\oiint\limits_{|\eta |=1}g(x+at\eta ,\tau )dS_{\eta }\Biggr|_{t=0}=g(x, \tau )\frac{1}{4\pi}\oiint\limits_{|\eta |=1}dS_{\eta}=g(x,\tau ).
\end{split}
\end{equation*}
\end{enumerate}
\end{proof}

Перейдем к решению задачи Коши
\begin{equation}
\left\{
  \begin{array}{ll}
  u_{tt}-a^2\Delta u=0,\ t>0,\, x\in \R^3\\
  \left. u\right|_{t=0}=0;\ \left. u_t\right|_{t=0}=u_1(x)
  \end{array}
\right.
\label{eq::5::cauchy}
\end{equation}
\begin{theorem}[Формула Пуассона-Кирхгофа]
Пусть $u_1(x)\in C^2(\R^3)$. Тогда 
\[
u(t,x)=\frac{1}{4\pi a^2t}\oiint\limits_{|\xi - x|=at}u_1(\xi )dS_{\xi} \quad \in \: C^2\, \{t\geq 0, x\in \R^3 \}
\]
является классическим решением задачи \eqref{eq::5::cauchy}
\end{theorem}
\begin{proof}
В силу леммы имеем $\left. u\right|_{t=0}=0;\ \left. u_t\right|_{t=0}=u_1$.

Т.к. $u_1\in C^2(\R^2),\ u\in C^2\,\{t\geq 0, x\in\R^3\}$, то сделаем замену переменной:
\[
u(t, x)=\frac{t}{4\pi}\oiint\limits_{|\eta |=1}u_1(x+at\eta )dS_{\eta}.
\]
Осталось только проверить, что $u$ удовлетворяет уравнению
\[
\Delta_xu(x, t)=\frac{t}{4\pi}\oiint\limits_{|\eta |=1}\Delta_{\xi}u_1(x+at\eta )dS_{\eta}=\frac{1}{4\pi a^2t}\oiint\limits_{|\xi - x|=at}\Delta_{\xi}u_1(\xi )dS_{\xi}.
\]
При $t>0$ (использовано $\vec{n}=\dfrac{\xi - x}{|\xi - x|}=\dfrac{\xi - x}{at}=\eta$)
\begin{equation*}
\begin{split}
&u_t(x,t)=\frac{1}{4\pi}\oiint\limits_{|\eta |=1}u_1(x+at\eta )dS_{\eta}+\frac{ta}{4\pi}\oiint\limits_{|\eta |=1}\sum\limits_{k=1}^3 \frac{\partial u_1}{\partial \xi_k}(x+at\eta )\eta_k\cdot dS_{\eta}=\\[0.75em]
&=\frac{u(x, t)}{t}+\frac{ta}{4\pi}\oiint\limits_{|\eta |=1}\sum\limits_{k=1}^3 \frac{\partial u_1}{\partial \xi_k}(\xi )n_k(\xi )\cdot dS_{\eta}=\frac{u(x, t)}{t}+\frac{1}{4\pi at}\oiint\limits_{|\xi - x |=at}\frac{\partial u_1}{\partial \vec{n}}(\xi )\cdot dS_{\xi}=\frac{u(x, t)}{t} + \frac{1}{4\pi at}I
\end{split}
\end{equation*}
Заметим, что
\[
\iiint\limits_{|\xi - x|<at}\Delta_{\xi}u_1(\xi)\, d\xi =\iiint\limits_{|\xi - x|<at}\mathrm{div} (\nabla u_1)\, d\xi =\oiint\limits_{|\xi - x|=at} (\nabla u_1,\, \vec{n})\, dS_{\xi }=\oiint\limits_{|\xi - x|=at}\frac{\partial u_1}{\partial\vec{n}}\, dS_{\xi }=I
\]
Тогда получим 
$$
u_t=\frac{u}{t}+\frac{1}{4\pi at}\iiint\limits_{|\xi - x|<at}\Delta_{\xi}u_1(\xi )d\xi \\
= \frac{u}{t}+\frac{1}{4\pi at}\int\limits_0^{at} \left[\oiint\limits_{|\xi - x|=\rho}\Delta_{\xi}u_1(\xi )dS_{\xi}\right]d\rho \\
= \frac{u}{t}+\frac{1}{4\pi at}\int\limits_0^{at}\phi (\rho )d\rho
$$

\begin{equation*}
\begin{split}
&u_{tt}(x, t)=\frac{\partial}{\partial t}\left[\frac{u}{t}+\frac{I}{4\pi at} \right]=\frac{u_t}{t}-\frac{u}{t^2}-\frac{I}{4\pi at^2}+\frac{I_t}{4\pi at}=\frac{u}{t\cdot t}+\frac{I}{4\pi at^2}-\frac{u}{t^2}-\frac{I}{4\pi at^2}+\frac{I_t}{4\pi at}=\\
&=\frac{I_t}{4\pi at}=\frac{1}{4\pi at}\frac{\partial}{\partial t}\int\limits_0^{at}\phi (\rho )d\rho = \frac{1}{4\pi at}a\phi(at)=\frac{1}{4\pi t}\oiint\limits_{|\xi - x|=at}\Delta_{\xi}u_1(\xi )dS_{\xi}.
\end{split}
\end{equation*}
Итак, $u(x, t)$ - классическое решение.
\end{proof}

Рассмотрим
\begin{equation}
\left\{
  \begin{array}{ll}
  u_{tt}-a^2\Delta u=0,\ t>0,\, x\in \R^3\\
  \left. u\right|_{t=0}=u_0(x);\ \left. u_t\right|_{t=0}=0,\ u_0\in C^3(\R^3)
  \end{array}
\right.
\label{eq::5::cauchy2}
\end{equation}

Введем $v(t, x)$:
\[
\left\{
  \begin{array}{ll}
  v_{tt}-a^2\Delta v=0,\ t>0,\, x\in \R^3\\
  \left. v\right|_{t=0}=0;\ \left. v_t\right|_{t=0}=u_0(x)
  \end{array}
\right.
\]
Эту задачу мы уже решили. Так как $u_0\in C^3(\R^3)$, имеем $v\in C^3\,\{t\geq 0, x\in\R^3\}$.
\begin{statement}
$u(x, t)\equiv v_t(x, t)\in C^2\,\{t\geq 0, x\in\R^3 \}$ дает решение задачи \eqref{eq::5::cauchy2}.
\end{statement}
\begin{proof}$\ $
\begin{enumerate}
\item $u_{tt} - a^2 \Delta u = \partial_t (v_{tt} - a^2 \Delta v) = \partial_t\: 0 = 0$
%$v_{tt}-a^2\Delta v=0 \Rightarrow v_{ttt}-a^2(\Delta v)_t=0\Rightarrow (v_t)_{tt}-a^2\Delta (v_t)=0\Rightarrow u_{tt}-a^2\Delta u=0$
\item $\left. u\right|_{t=0}=\left. v_t\right|_{t=0}=u_0(x)$
\item $\while{u_t}{t=0} = \while{v_{tt}}{t=0}=\while{a^2 \Delta v}{t=0}=0$ (на гиперплоскости $t=0\colon\ \while{v}{t=0} \equiv 0\ \Rightarrow\ \while{\pd{v}{x_i}}{t=0} \equiv 0\ \Rightarrow\ \while{\pdd{v}{x_i}}{t=0} \equiv 0.\;$ При этом $v_t$ может быть ненулевой.)
\end{enumerate}
\end{proof}
Мы доказали следующую теорему:
\begin{theorem}
Функция $u(t,x)=\dfrac{\partial}{\partial t}\left[\dfrac{1}{4\pi a^2t}\displaystyle\oiint\limits_{|\xi -x|=at}u_0(\xi )dS_{\xi } \right],\ t\geq 0,x\in\R^3$, где $u_0\in C^3(\R^3)$, \\[1em]
является классическим решением задачи \eqref{eq::5::cauchy2}. 
\end{theorem}

\end{document}