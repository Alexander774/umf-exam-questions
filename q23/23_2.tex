\begin{theorem}
Пусть $u(x)$ -- гармоническая функция в окрестности бесконечности $|x| > R$ и $u(x) \rightarrow 0, x \rightarrow \infty$. Тогда $u(x) = O\brs{\frac{1}{|x|}}$ и $D^\alpha u(x) = O\brs{\frac{1}{|x|^{1+|\alpha|}}}$ при $x\rightarrow \infty$
\end{theorem}
\begin{proof}
Применим к нашей функции прямое преобразование Кельвина: $u^*(y) = \frac{R}{|y|}u\brs{\frac{R^2}{|y|^2}y}$. Эта функция гармоническая в проколотой окрестности нуля: $0 < |y| < R$.\\
Далее, $|y|u^*(y) = R u\brs{\frac{R^2}{|y|^2}y} \rightarrow 0, y \rightarrow 0$ (аргумент $\rightarrow \infty$) $\Rightarrow u^*(y) = o \brs{\frac{1}{|y|}}, y \rightarrow 0$. 
\begin{itemize}
\item Воспользуемся теоремой об устранимой точке и доопределим $u^*(y)$ в нуле. Теперь $u^*(y)$ -- гармоническая в $|y| < R \Rightarrow$ в $|y| \leq \frac{R}{2}$ есть непрерывность вплоть до границы любых производных $\Rightarrow \forall \alpha \exists M_\alpha: \abs{D^\alpha u^*(y)} \leq M_\alpha, \forall y : |y| \leq \frac{R}{2}$
\item Пусть $|x| \geq 2R, y = x^* \Rightarrow |x^*| = |y| \leq \frac{R}{2}$. Тогда $u(x) = u\brs{\frac{R^2}{|y|^2}y} = \frac{R}{|x||y|} u\brs{\frac{R^2}{|y|^2}y}  = \frac{R}{|x|} u^*(y)$. Можем оценить: $|u(x)| \leq \frac{R}{|x|} M_0 \Rightarrow u(x) = O\brs{\frac{1}{|x|}}, x \rightarrow \infty$.
\item Возьмем теперь $\alpha = (1,0,0)$: 
\begin{align*}
\pd{u(x)}{x_1} &= D^\alpha u(x) = \pd{}{x_1} \brs{\frac{R}{|x|} u^*(y)} = - \frac{R}{|x|^3} x_1 u^*(y) + \frac{R}{|x|} \sum_{k=1}^3 \pd{u^*(y)}{y_k} \pd{y_k}{x_1} =\\ 
&=- \frac{R}{|x|^3} x_1 u^*(y) + \frac{R^3}{|x|^3} \sum_{k=1}^3 \pd{u^*(y)}{y_k} \sbrs{\delta_k^1 - 2 \frac{x_1 x_k}{|x|^2}}
\end{align*}
Оценка: $$\abs{\pd{u(x)}{x_1}} \leq R \frac{x_1}{|x|} \frac{1}{|x|^2} \underbrace{\abs{u^*(y)}}_{\leq M_0} + \frac{R^3}{|x|^3}\sum_{k=1}^3 \underbrace{\abs{\pd{u^*(y)}{y_k}}}_{\leq M_{(1,0,0)}} \sbrs{\delta_k^1 + 2 \frac{|x_1| \cdot |x_k|}{|x|^2}} \leq \frac{C}{|x|^2}$$
Итак, $\pd{u(x)}{x_1} = O \brs{\frac{1}{|x|^2}}$. Аналогично, по индукции, и для других производных.
\end{itemize}
\end{proof}
Постановка внешних задач.
\begin{definition}
Область $\Omega \subset \R^3$ называется внешней, если $\R^3 \backslash \overline{\Omega} = \Omega_1$ -- ограниченная область в $\R^3$ 
\end{definition}
\begin{definition}
Внешнюю область в $\Omega$ будем называть внешней областью с гладкой (кусочно-гладкой) границей, если $\Omega_1 = \R^3 \backslash \overline{\Omega}$ -- область с (кусочно-гладкой) границей.
\end{definition}